\documentclass{article}

\usepackage{amsmath}
\usepackage{amsfonts}
\usepackage{amssymb}
\usepackage{commath}

\usepackage[a4paper,margin=1.8cm,footskip=.5cm]{geometry}
\usepackage{hyperref}

% this document is supposed to compiled using latexmk; e.g.
% latexmk -pdf euler_equations.tex

\title{Minimal information about Euler equations}
\author{Pierre Kestener}

\date{\today}

\begin{document}

\maketitle

\begin{abstract}
Just gathering important equations about the compressible Euler system of equations in 1D, 2D and 3D space. In particular, we give all the flux Jacobian matrix, and the corresponding eigen decomposition in conservative and primitive variables in the case of perfect gas EOS.
\end{abstract}

\section{Compressible Euler equations}

Let's define the vector of conservative variables $U=(\rho, \rho u, \rho v, \rho w, E)$ and the corresponding vector of primitive variables $W=(\rho, u, v, w, p)$, where $\rho$ is the fluid density, $u,v,w$ are the three cartesian components of the velocity vector field, $E$ is the total energy per unit volume and $p$ is the pressure.

Speed of sound $c=\sqrt{\left(\frac{\partial p}{\partial \rho}\right)_s} = \sqrt{\frac{\gamma p}{\rho}}$

\subsection{Euler system in conservative variables}

\subsubsection*{1D Euler system in conservative variables}
The compressible Euler system reads in conservative variables:\\
\begin{equation}
  \begin{array}{ccccc}
    \partial_t \rho & + & \partial_x(\rho u) & = & 0,\\
    \partial_t (\rho u) & + & \partial_x(\rho u^2+p) & = & 0,\\
    \partial_t E & + & \partial_x (u(E+p)) & = & 0,\\
  \end{array}
\end{equation}

or in short notations:
\begin{equation*}
  \partial_t \mathbf{U} + \partial_x \mathbf{F(U)} = \mathbf{0},
\end{equation*}

where the flux function is defined by
\begin{equation}
  \mathbf{F(U)} = \left [
  \begin{array}{c}
    \rho u \\
    \rho u^2 + p \\
    u (E + p)
  \end{array} \right]
\end{equation}

\begin{itemize}
\item total (internal + kinetic) energy per unit volume $E = \rho \left( e + \frac{1}{2} u^2 \right) = \frac{p}{\gamma-1} + \frac{1}{2} \rho u^2$,
\item specific (per mass unit) internal energy $e=\frac{p}{(\gamma-1)\rho}$.
\item total enthalpy $H = E + p$ per unit volume
\end{itemize}


\subsubsection*{2D Euler system in conservative variables}
The compressible Euler system reads in conservative variables:\\
\begin{equation}
  \begin{array}{ccccccc}
    \partial_t \rho & + & \partial_x(\rho u) & + & \partial_y(\rho v) & = & 0,\\
    \partial_t (\rho u) & + & \partial_x(\rho u^2+p) & + & \partial_y(\rho v u) & = & 0,\\
    \partial_t (\rho v) & + & \partial_x(\rho u v) & + & \partial_y(\rho v^2+p) & = & 0,\\
    \partial_t E & + & \partial_x (u(E+p)) & + & \partial_y (v(E+p)) & = & 0,\\
  \end{array}
\end{equation}

or in short notations:
\begin{equation*}
  \partial_t \mathbf{U} + \partial_x \mathbf{F(U)} + \partial_y \mathbf{G(U)} = \mathbf{0},
\end{equation*}

where the flux functions are defined by
\begin{equation}
  \mathbf{F(U)} = \left [
  \begin{array}{c}
    \rho u \\
    \rho u^2 + p \\
    \rho u v \\
    u (E + p)
  \end{array} \right]
  \;\; ,
  \mathbf{G(U)} = \left [
  \begin{array}{c}
    \rho v \\
    \rho v u \\
    \rho v^2 + p \\
    v (E + p)
  \end{array} \right]
\end{equation}

Total (internal + kinetic) energy $E = \rho \left( e + \frac{1}{2} (u^2+v^2) \right) = \frac{p}{\gamma-1} + \frac{1}{2} \rho (u^2+v^2)$.

\subsubsection*{3D Euler system in conservative variables}
The compressible Euler system reads in conservative variables:\\
\begin{equation}
  \begin{array}{ccccccccc}
    \partial_t \rho & + & \partial_x(\rho u) & + & \partial_y(\rho v) & + & \partial_z(\rho w) & = & 0,\\
    \partial_t (\rho u) & + & \partial_x(\rho u^2+p) & + & \partial_y(\rho v u) & + & \partial_z(\rho w u) & = & 0,\\
    \partial_t (\rho v) & + & \partial_x(\rho u v) & + & \partial_y(\rho v^2+p) & + & \partial_z(\rho w v) & = & 0,\\
    \partial_t (\rho w) & + & \partial_x(\rho u w) & + & \partial_y(v w) & + & \partial_z(\rho w^2+p) & = & 0,\\
    \partial_t E & + & \partial_x (u(E+p)) & + & \partial_y (v(E+p)) & + & \partial_z (w(E+p)) & = & 0,\\
  \end{array}
\end{equation}

or in short notations:
\begin{equation*}
  \partial_t \mathbf{U} + \partial_x \mathbf{F(U)} + \partial_y \mathbf{G(U)} + \partial_z \mathbf{H(U)} = \mathbf{0},
\end{equation*}

where the flux functions are defined by
\begin{equation}
  \mathbf{F(U)} = \left [
  \begin{array}{c}
    \rho u \\
    \rho u^2 + p \\
    \rho u v \\
    \rho u w \\
    u (E + p)
  \end{array} \right]
  \;\; ,
  \mathbf{G(U)} = \left [
  \begin{array}{c}
    \rho v \\
    \rho v u \\
    \rho v^2 + p \\
    \rho v w \\
    v (E + p)
  \end{array} \right]
  \;\; ,
  \mathbf{H(U)} = \left [
  \begin{array}{c}
    \rho w \\
    \rho w u \\
    \rho w v \\
    \rho w^2 + p \\
    w (E + p)
  \end{array} \right]
\end{equation}

Total (internal + kinetic) energy $E = \rho \left( e + \frac{1}{2} (u^2+v^2+w^2) \right) = \frac{p}{\gamma-1} + \frac{1}{2} \rho (u^2+v^2+w^2)$.

\subsection{Euler system in primitive variables}

\subsubsection*{1D Euler system in primitive variables}
The compressible Euler system reads in primitive variables~\cite{toro} (section 3.2.3):\\

\begin{equation}
  \begin{array}{ccccc}
    \rho_t & + & u\rho_x + \rho u_x   & = & 0,\\
    u_t    & + & u u_x   + p_x/\rho   & = & 0,\\
    p_t    & + & \gamma p u_x + u p_x & = & 0,
  \end{array}
\end{equation}

where the subscript variable designates the variable used in the partial derivative, e.g $\rho_t = \partial_t \rho$

\begin{equation}
  \partial_t \mathbf{W} + \mathbf{A(W)} \partial_x \mathbf{W} = \mathbf{0}
\end{equation}

with
\begin{equation}
  \mathbf{A(W)} = \left[
    \begin{array}{c c c c}
      u & \rho     & 0 \\
      0 & u        & 1/\rho\\
      0 & \gamma p & u \\
    \end{array}
  \right]
\end{equation}

$A(W)$ has 3 eigenvalues: $\lambda_{-}=u-c$, $\lambda_{0}=u$ and $\lambda_{+}=u+c$, where $c=\sqrt{\frac{\gamma p}{\rho}}$ is the speed of sound. 

A right column-eigenvectors basis:
\begin{equation}
  R^{(0)}=\left[
    \begin{array}{c}
      1\\
      -c/\rho\\
      c^2
    \end{array} \right],
  \;\;
  R^{(1)}=\left[
    \begin{array}{c}
      1\\
      0\\
      0
    \end{array} \right],
  \;\;
  R^{(2)}=\left[
    \begin{array}{c}
      1\\
      c/\rho\\
      c^2
    \end{array} \right],
\end{equation}

A left line-eigenvector basis:
\begin{equation}
  L^{(0)}=\left[0, -\frac{\rho}{2c}, \frac{1}{2c^2}\right],\;\;\\
  L^{(1)}=\left[1, 0, -\frac{1}{c^2} \right],\;\;\\
  L^{(2)}=\left[0, \frac{\rho}{2c}, \frac{1}{2c^2}\right],
\end{equation}

Eigen identity : $\Lambda = L A(W) R$, where the matrix $L$ and $R$ are formed with the corresponding eigenvectors:
\begin{equation}
  \Lambda = \left[
    \begin{array}{ccc}
      u-c& 0 & 0\\
      0  & u & 0\\
      0  & 0 & u+c
    \end{array}
  \right],\;
  L=\left[
    \begin{array}{ccc}
      0 & -\frac{\rho}{2c} & \frac{1}{2c^2}\\
      1 & 0 & -\frac{1}{c^2}\\
      0 & \frac{\rho}{2c} & \frac{1}{2c^2}
    \end{array}
  \right],\;
  A(W)=\left[
    \begin{array}{c c c c}
      u & \rho     & 0 \\
      0 & u        & 1/\rho\\
      0 & \gamma p & u \\
    \end{array}
  \right],\;
  R=\left[
    \begin{array}{ccc}
      1 & 1 & 1\\
      -c/\rho & 0 & c/\rho\\
      c^2 & 0 & c^2
    \end{array}
  \right].
\end{equation}

You can also check that $L$ is the the matrix inverse of $R$: $L = R^{-1}$.

\subsubsection*{2D Euler system in primitive variables}
The compressible Euler system reads in primitive variables~\cite{toro} (section 3.2.3):\\

\begin{equation}
  \begin{array}{ccccccc}
    \rho_t & + & u\rho_x+\rho u_x & + & v\rho_y+\rho u_y & = &0,\\
    u_t    & + & u u_x + p_x/\rho & + & v u_y            & = &0,\\
    v_t    & + & u v_x            & + & v v_y + p_y/\rho & = &0,\\
    p_t    & + & \gamma p u_x + u p_x & + & \gamma p v_y + v p_y & = &0,
  \end{array}
\end{equation}

where the subscript variable designates the variable used in the partial derivative, e.g $\rho_t = \partial_t \rho$

\begin{equation}
  \partial_t \mathbf{W} + \mathbf{A(W)} \partial_x \mathbf{W} + \mathbf{B(W)} \partial_y \mathbf{W} = \mathbf{0}
\end{equation}

with
\begin{equation}
  \mathbf{A(W)} = \left[
    \begin{array}{c c c c}
      u & \rho     & 0 & 0 \\
      0 & u        & 0 & 1/\rho\\
      0 & 0        & u & 0 \\
      0 & \gamma p & 0 & u \\
    \end{array}
  \right],
  \;\;
  \mathbf{B(W)} = \left[
    \begin{array}{c c c c}
      v & 0 & \rho     & 0 \\
      0 & v & 0        & 0\\
      0 & 0 & v        & 1/\rho \\
      0 & 0 & \gamma p & v \\
    \end{array}
  \right],
\end{equation}

$A(W)$ and B(W) have 4 eigenvalues: $\lambda_{-}=u-c$, $\lambda_{0}=u$ (2 times) and $\lambda_{+}=u+c$, where $c=\sqrt{\frac{\gamma p}{\rho}}$ is the speed of sound. 

Eigen decompostions are: $\Lambda = L_A A(W) R_A$ and $\Lambda = L_B B(W) R_B$ with

% A
\begin{equation}
  \Lambda = \left[
    \begin{array}{cccc}
      u-c& 0 & 0 & 0\\
      0  & u & 0 & 0\\
      0  & 0 & u & 0\\
      0  & 0 & 0 & u+c
    \end{array}
  \right],\;\;
  L_A = \left[
    \begin{array}{cccc}
      0                  & -\frac{\rho}{2c} & 0                & \frac{1}{2c^2}\\
      \frac{1}{1-\rho v} & 0                & -\frac{\rho}{1-\rho v} & -\frac{1}{c^2 (1-\rho v)}\\
     -\frac{v}{1-\rho v} & 0                & \frac{1}{1-\rho v}     &  \frac{v}{c^2 (1-\rho v)}\\
      0 & \frac{\rho}{2c}  & 0                & \frac{1}{2c^2}
    \end{array}
  \right],\;\;
  R_A = \left[
    \begin{array}{cccc}
      1       & 1 & \rho & 1\\
      -c/\rho & 0 & 0    & c/\rho\\
      0       & v & 1    & 0\\
      c^2     & 0 & 0    & c^2
    \end{array}
  \right],\;\;
\end{equation}

% B
\begin{equation}
  \Lambda = \left[
    \begin{array}{cccc}
      u-c& 0 & 0 & 0\\
      0  & u & 0 & 0\\
      0  & 0 & u & 0\\
      0  & 0 & 0 & u+c
    \end{array}
  \right],\;\;
  L_B = \left[
    \begin{array}{cccc}
      0                  & -\frac{\rho}{2c} & 0                & \frac{1}{2c^2}\\
      \frac{1}{1-\rho v} & 0                & -\frac{\rho}{1-\rho v} & -\frac{1}{c^2 (1-\rho v)}\\
     -\frac{v}{1-\rho v} & 0                & \frac{1}{1-\rho v}     &  \frac{v}{c^2 (1-\rho v)}\\
      0 & \frac{\rho}{2c}  & 0                & \frac{1}{2c^2}
    \end{array}
  \right],\;\;
  R_B = \left[
    \begin{array}{cccc}
      1       & 1 & \rho & 1\\
      0       & 1 & u    & 0\\
      -c/\rho & 0 & 0    & c/\rho\\
      c^2     & 0 & 0    & c^2
    \end{array}
  \right],\;\;
\end{equation}



\subsubsection*{3D Euler system in primitive variables}
The compressible Euler system reads in primitive variables~\cite{toro} (section 3.2.3):\\

\begin{equation}
  \begin{array}{ccccccccc}
    \rho_t & + & u\rho_x+\rho u_x & + & v\rho_y+\rho u_y & + & w\rho_z+\rho u_z & = &0,\\
    u_t    & + & u u_x + p_x/\rho & + & v u_y            & + & w u_z            & = &0,\\
    v_t    & + & u v_x            & + & v v_y + p_y/\rho & + & w v_z            & = &0,\\
    w_t    & + & u w_x            & + & v w_y            & + & w w_z + p_z/\rho & = &0,\\
    p_t    & + & \gamma p u_x + u p_x & + & \gamma p v_y + v p_y & + & \gamma p w_z + w p_z & = &0,
  \end{array}
\end{equation}

where the subscript variable designates the variable used in the partial derivative, e.g $\rho_t = \partial_t \rho$

\begin{equation}
  \partial_t \mathbf{W} + \mathbf{A(W)} \partial_x \mathbf{W} + \mathbf{B(W)} \partial_y \mathbf{W} + \mathbf{C(W)} \partial_z \mathbf{W} = \mathbf{0}
\end{equation}

with
\begin{equation}
  \mathbf{A(W)} = \left[
    \begin{array}{c c c c c}
      u & \rho     & 0 & 0 & 0 \\
      0 & u        & 0 & 0 & 1/\rho\\
      0 & 0        & u & 0 & 0 \\
      0 & 0        & 0 & u & 0 \\
      0 & \gamma p & 0 & 0 & u \\
    \end{array}
  \right],
  \;\;
  \mathbf{B(W)} = \left[
    \begin{array}{c c c c c}
      v & 0 & \rho     & 0 & 0 \\
      0 & v & 0        & 0 & 0\\
      0 & 0 & v        & 0 & 1/\rho \\
      0 & 0 & 0        & v & 0 \\
      0 & 0 & \gamma p & 0 & v \\
    \end{array}
  \right],
  \;\;
  \mathbf{C(W)} = \left[
    \begin{array}{c c c c c}
      w & 0 & 0 & \rho     & 0 \\
      0 & w & 0 & 0        & 0\\
      0 & 0 & w & 0        & 0 \\
      0 & 0 & 0 & w        & 1/\rho \\
      0 & 0 & 0 & \gamma p & w \\
    \end{array}
  \right].
\end{equation}


\subsection{Quasi-linear form in conservative variable}

The compressible Euler system can be rewritten in a quasi-linear form:

\begin{equation}
  \partial_t \mathbf{U} + \mathbf{A(U)} \partial_x \mathbf{U} + \mathbf{B(U)} \partial_y \mathbf{U} + \mathbf{C(U)} \partial_z \mathbf{U} = \mathbf{0},
\end{equation}

where $A(U)=\frac{\partial F}{\partial U}$, $B(U)=\frac{\partial G}{\partial U}$, $C(U)=\frac{\partial H}{\partial U}$ are flux Jacobian matrices.

When using the following notation for the conservative variables vector $U=[u_1, u_2, u_3, u_4, u_5]$, and for the flux vector $F=[f_1, f_2, f_3, f_4, f_5]$, the x direction flux Jacobian matrix $A(U)$ reads

\begin{equation}
  A(U)=\frac{\partial F}{\partial U} = \left[ \partial f_i/\partial u_j \right] = \left [
    \begin{array}{c c c}
      \partial f_1/\partial u_1 & \dots & \partial f_1/\partial u_5\\
      \partial f_2/\partial u_1 & \dots & \partial f_2/\partial u_5\\
      \vdots & \ddots & \vdots\\
      \partial f_5/\partial u_1 & \dots & \partial f_5/\partial u_5\\
    \end{array} \right ]
  = \left[
    \begin{array}{c c c c c}
      
    \end{array} \right ]
\end{equation}

\begin{thebibliography}{9}
\bibitem[Toro]{toro} \emph{Riemann solvers and numerical methods for fluid dynamics. A practical introduction.}, E.F. Toro , Springer, 2nd edition, 1999.
\bibitem[Zingale]{zingale} \emph{Notes on the Euler equations}, \url{http://bender.astro.sunysb.edu/hydro_by_example/compressible/Euler.pdf}
\end{thebibliography}

\end{document}
